\documentclass[transmag]{IEEEtran}
\usepackage{latexsym}
\usepackage{graphicx}
\usepackage{amsfonts,amssymb,amsmath}
\usepackage{hyperref}
\graphicspath{ {./images/} }
\renewcommand\IEEEkeywordsname{Keywords}
\usepackage{pdfpages}
\def\BibTeX{{\rm B\kern-.05em{\sc i\kern-.025em b}\kern-.08em T\kern-.1667em\lower.7ex\hbox{E}\kern-.125emX}}
\usepackage{float}
\usepackage{aliascnt}
\newaliascnt{eqfloat}{equation}
\newfloat{eqfloat}{h}{eqflts}
\floatname{eqfloat}{Equation}
\usepackage{cancel}

\begin{document}

\title{\textsc{Robot Manipulator Design Assignment}}

\clearpage\thispagestyle{empty}

\author{Souto T.L.

\\
\\
\\

\begin{centering}
\vspace{20mm}
\includegraphics[scale=0.25]{massey-png}
\end{centering}



\thanks{This paper is a Individual design Project and It is part of the second assignment of the course Master of engineering - Mechatronics at Massey university, Auckland.

Thiago Lima Souto is a student register under the number 19044686 at Massey university. Questions, comments or communications can be addressed via email \color{blue}\href{mailto:thiago.souto@yahoo.com.br}{thiago.souto@yahoo.com.br}}}


\IEEEtitleabstractindextext{\begin{abstract}
In this paper is reported the design of a robotic manipulator with a fixed platform in a flat surface, It's able to be integrated with a robotic gripper also designed. The objective of this robotic system is to pick an object from a shelf or from the wall and place it onto a horizontal surface.
Several tools were used to accomplish the objective of this project. For the calculations of forward and inverse kinematics Python programming language and the Pycharm IDE(Integrated Development Environment) were used, for modelling the robotic system SolidWorks, to simulate Mathworks Simulink and OpenModelica were chosen.

\end{abstract}



\begin{IEEEkeywords}
robotic systems, forward kinematics, inverse kinematics 
\end{IEEEkeywords}
}

\maketitle
\thispagestyle{empty}

\clearpage
\newpage

\clearpage\thispagestyle{empty}
\onecolumn
\tableofcontents

\clearpage
\newpage

\twocolumn

\section{INTRODUCTION}

Industrial robot systems as well as computer-aided design and manufacturing (CAD and CAM) are leading the industrial automation. \cite{ref1}

The mechanical manipulator is the most important form of the industrial robot and the localization of objects in the three-dimensional space is one of the most important aspect of the mechanical manipulator. Links, parts, tools, other objects on the manipulator environment and the motion of these objects are the subjects of study of Kinematics,  as well as all the geometrical and time-based properties of the motion, with no regards to the forces applied that causes it.

The two basic problems in the study of mechanical manipulation are forward and inverse kinematics, the first computes the position and orientation of the end-effector on the manipulator and the second calculates all possible sets of joint angles that could be used  a given position and orientation. 

Nowadays, CAD and CAM advanced software's are of easy access and used to design, simulate and calculate all that is necessary for modern robot design.

The main objective of this assessment is to design a robotic manipulator with a fixed platform and flat surface, that is able to be integrated with a robot gripper for picking an object vertical wall/shelf and placing it onto a horizontal surface. 

To accomplish this objective a robot system is proposed after this introduction followed by the manipulator and other components design. The forward and inverse kinematics of the robot system are studied with manual calculations as well as computed calculations. A Model with correct dimensions and a simulation of the proposed robot are made using Solidworks and OpenModelica. Finally, the results are discussed and the report is concluded.


 

\section{Robot System Initial Proposal}

Propose a robotic system that consists of a manipulator, a gripper, sensing, motors, controllers, and control methodologies, which is able to realise the above required functions. (3 marks)


\section{Manipulator Design}
\subsection{Robot Arm}
When you submit your final version (after your paper has been accepted), 
print it in two-column format, including figures and tables. You must also 
send your final manuscript on a disk, via e-mail, or through a Web 
manuscript submission system as directed by the society contact. You may use 
\emph{Zip} or CD-ROM disks for large files, or compress files using \emph{Compress, Pkzip, Stuffit,} or \emph{Gzip.} 


\subsection{Robot Gripper}
Also, send a sheet of paper or PDF with complete contact information for all 
authors. Include full mailing addresses, telephone numbers, fax numbers, and 
e-mail addresses. This information will be used to send each author a 
complimentary copy of the journal in which the paper appears. In addition, 
designate one author as the ``corresponding author.'' This is the author to 
whom proofs of the paper will be sent. Proofs are sent to the corresponding 
author only.

\subsection{Entire Model}
Create a model of the robot manipulator with the correct dimensions. Using any method, software and hardware you are familiar with to simulate the movement (pick and place) of the robot manipulator. (3 marks)

\section{Other Components}
Also, send a sheet of paper or PDF with complete contact information for all 
authors. Include full mailing addresses, telephone numbers, fax numbers, and 
e-mail addresses.



\subsection{Sensors}

To know the angular position of the joints absolute encoders shall be the choice because they give the actual angular position, a unique identification of an angle. The incremental encoders would detect the changes but in relation to some Datum. \cite{ref3}

So with the absolute encoders we can track $\theta_1,\theta_2$, $\theta_3$ and $\theta_4$ and rearrange the links accordingly with the joints angles.

Also a loading cell can be used on the toll to sense the amount of pressure to set a trigger to avoid damage on the object.

\subsection{Motors}

Most charts graphs and tables are one column wide (3 1/2 inches or 21 picas) 
or two-column width (7 1/16 inches, 43 picas wide). We recommend that you 
avoid sizing figures less than one column wide, as extreme enlargements may 
distort your images and result in poor reproduction. Therefore, it is better 
if the image is slightly larger, as a minor reduction in size should not 
have an adverse affect the quality of the image. 

\subsection{Controllers}

The final printed size of an author photograph is exactly 
1 inch wide by 1 1/4 inches long (6 picas~$\times$~7 1/2 picas). Please 
ensure that the author photographs you submit are proportioned similarly. If 
the author's photograph does not appear at the end of the paper, then please 
size it so that it is proportional to the standard size of 1 9/16 inches 
wide by 
2 inches long (9 1/2 picas~$\times$~12 picas). JPEG files are only 
accepted for author photos.

\subsection{Control Methodology}

IEEE accepts color graphics in the following formats: EPS, PS, TIFF, Word, 
PowerPoint, Excel, and PDF. The resolution of a RGB color TIFF file should 
be 400 dpi. 

When sending color graphics, please supply a high quality hard copy or PDF 
proof of each image. If we cannot achieve a satisfactory color match using 
the electronic version of your files, we will have your hard copy scanned. 
Any of the files types you provide will be converted to RGB color EPS files. 


\section{Forward Kinematic}
* Design a robot manipulator that is able to position the centre of a robot gripper to a desired location. 
* Define the coordinate systems for robot manipulator joints, specify the size of each link, draw the symbolic representation of the robot manipulator, and identify the DH parameters. 
* Obtain the kinematic equations relating to the centre of the end-effector position and orientation. (5 marks)

\begin{figure}
\centerline{\includegraphics[width=3.5in]{./images/Axis}}
\caption{$\hat Z, \hat Y, \hat X axis, \theta_1, \theta_2, \theta_3, l1, l2, l3 and l4 $\label{fig13}}
\end{figure}

The forward kinematics calculations were confirmed by a python programming code that can be found in the Appendix \ref{appendix-A}. 

For the forward kinematics a class called "FowardKinematics" was created, this class has two main methods involved on the calculations, the rotation and the translation functions for the $\hat Z$ and $\hat X$ axis. The parameters for these methods are extracted from the Denavit–Hartenberg parameters table at \ref{eq1}, the coordinate systems and also the basic frames $\{B\}$, $\{W\}$ and $\{T\}$, Base, Wrist and Tools respectively are identified on the Figure \ref{fig1}. The size of the links are $l_1 = 220$, $l_2 = 500$, $l_3 = 500$, $l_4 = 150$.

The rotation method receives an argument $self$ and $\theta_i$ for the rotation on the $\hat Z$ axis and $\alpha_{i-1}$ for $\hat X$. The $self$ argument is what makes this a method and not just a plain function, this is filled in automatically, when we call this method on the object. So we'll just provide one argument, and the fact that it's being called on the method will provide the first argument, self. It will then build a $sympy$ symbolic matrix and passes the $self$ argument to the method to be put in place, if no arguments are passed default values will be put in place as specified in the key word arguments ($\**$ $kwargs$) on the $\_\_init\_\_$ function. A Matrix is then returned after calling the $.evalf()$ function to evaluate.

Like in the rotation method the translation method receives a argument $d_i$ and $a_{i-1}$ to return a matrix that translates in $\hat Z$ and $\hat X$ axis respectively. This class is also detailed in the Appendix \ref{appendix-A}.  

The objective of the forward kinematics is to provide a kinematics equation relating the end-effector orientation and position. This is done by finding the 




\begin{eqfloat}
\begin{equation}
\begin{tabular}{c|c|c|c|c}
$i$  & $\alpha_{i-1}$ & $a_{i-1}$ & $d_i$        & $\theta_i$ \\
\hline
1    &  0             &  0         &  $l_1$ & $\theta_1$ \\
2    &  $90^o$        &  0         &  0     & $\theta_2$ \\
3    &  0             &  $l_2$     &  0     & $\theta_3$ \\
4    &  0             &  $l_3$     &  0     & $\theta_4$ \\
5    &  0             &  $l_4$     &  0     & 0 \\
\end{tabular}
\end{equation}

\begin{center}
Denavit–Hartenberg parameters
\end{center}




\begin{equation}
^0_1T =
\begin{bmatrix}
cos\theta_1 & -sin\theta_1 & 0 & 0   \\
sin\theta_1 & cos\theta_1  & 0 & 0   \\
0           & 0            & 1 & l_1 \\
0           & 0            & 0 & 1   \\
\end{bmatrix}
\end{equation}


\begin{equation}
^1_2T =
\begin{bmatrix}
cos\theta_2 & -sin\theta_2 & 0      & 0   \\
-0.448 sin\theta_2     & -0.448 cos\theta_2      & -0.893 & 0   \\
0.893 sin\theta_2      & 0.893 cos\theta_2       & -0.448 & 0   \\
0           & 0            & 0      & 1   \\
\end{bmatrix}
\end{equation}




\begin{equation}
^2_3T =
\begin{bmatrix}
cos\theta_3 & -sin\theta_3 & 0      & l_2  \\
sin\theta_3 & cos\theta_3  & 0      & 0   \\
0           & 0            & 1      & 0   \\
0           & 0            & 0      & 1   \\
\end{bmatrix}
\end{equation}



\begin{equation}
^3_4T =
\begin{bmatrix}
cos\theta_4 & -sin\theta_4 & 0      & l_3  \\
sin\theta_4 & cos\theta_4  & 0      & 0   \\
0           & 0            & 1      & 0   \\
0           & 0            & 0      & 1   \\
\end{bmatrix}
\end{equation}



\begin{equation}
^4_5T =
\begin{bmatrix}
1 & 0 & 0 & l_4  \\
0 & 1 & 0 & 0   \\
0 & 0 & 1 & 0   \\
0 & 0 & 0 & 1   \\
\end{bmatrix}
\end{equation}

\begin{equation}
^0_5T = ^0_1T  ^1_2T  ^2_3T  ^3_4T  ^4_5T => ^0 \cancel{_1T  ^1_2T  ^2_3T  ^3_4T  ^4} _5T = ^0_5T
\end{equation}

\end{eqfloat}

\begin{figure*}
\begin{center}
$^0_5T = $
\end{center}
\begin{equation}
\begin{bmatrix}
\begin{matrix}
0.724 cos(-\theta_1 + \theta_2 + \theta_3) + \\ 
0.275 cos(\theta_1 + \theta_2 + \theta_3)
\end{matrix}
& 
\begin{matrix}
-0.724 sin(-\theta_1 + \theta_2 + \theta_3) - \\
0.275 sin(\theta_1 + \theta_2 + \theta_3)
\end{matrix}
& 

0.893 sin(\theta_1)

& 
\begin{matrix}
224.036 sin(\theta_1) sin(\theta_2) +\\
500 sin(\theta_1) sin(\theta_2) + \\
434.42 cos(-\theta_1 + \theta_2 + \theta_3) + \\
165.57 cos(\theta_1 + \theta_2 + \theta_3)
\end{matrix} 
\\
&&&&
\\
\begin{matrix}
-0.724 sin(-\theta_1 + \theta_2 + \theta_3) + \\ 
0.275 sin(\theta_1 + \theta_2 + \theta_3)
\end{matrix}
& 
\begin{matrix}
-0.724 cos(-\theta_1 + \theta_2 + \theta_3) + \\
0.275 cos(\theta_1 + \theta_2 + \theta_3)
\end{matrix}
& 

-0.893 cos(\theta_1)

& 
\begin{matrix}
500 sin(\theta_1) cos(\theta_2) +\\
-224.03 sin(\theta_2) cos(\theta_1) + \\
-434.42 sin(-\theta_1 + \theta_2 + \theta_3) + \\
165.57 sin(\theta_1 + \theta_2 + \theta_3)
\end{matrix} 
\\
&&&&
\\
0.893 sin(\theta_1 + \theta_2)      & 0.893 cos(\theta_1 + \theta_2)            & -0.448      
&
\begin{matrix}
446.9 sin(\theta_2) + \\
536.39 sin(\theta_2 + \theta_3) +220
\end{matrix}
\\
&&&&
\\
0           & 0            & 0      & 1   \\
\end{bmatrix}
\end{equation}
\end{figure*}








\section{Inverse Kinematics}

Obtain the inverse kinematics equations to find the joint displacements leading the centre of the end-effector from a vertical position to a horizontal position with correct orientation. (5 marks)

\section{System Simulation}

OpenModelica is currently the most complete opensource Modelica- and FMI-based modeling, simulation, optimization, and model-based development environment. Its long-term development is supported by a non-profit organization – the Open Source Modelica Consortium (OSMC). \cite{ref2}
This system was chosen because of the opensource aspect, since Mathworks Simulink requires a paid plugin to connect the Solidworks model.


\begin{figure}
\centerline{\includegraphics[width=3.5in]{./images/openmodelica}}
\caption{Openmodelica, modeling, simulation, optimization, and model-based development environment.\label{fig1}}
\end{figure}


For the simulation, information from the CAD simulator (SolidWorks) regarding to mass, 
center of mass and moments of inertia, were confronted with the Python simulation and was 
consistent as shown on Figure \ref{fig2}, other information about the model is also provided but not included on the link properties at OpenModelica, like density, volume, surface area, among others. The parameters necessary for the model was the length, mass and center of mass, as well as the inertia tensors, as shown on Figure \ref{fig3}. The programming documentations and results can be found on Appendix A


\begin{figure}
\centerline{\includegraphics[width=3.5in]{./images/MassProperties}}
\caption{Mass, center of mass and moments of inertia used on the simulation from link1 - SolidWorks.\label{fig2}}
\end{figure}


\begin{figure}
\centerline{\includegraphics[width=3.5in]{./images/Link1Parameters}}
\caption{OpenModelica, link1 parameters\label{fig3}}
\end{figure}
 


By using the useAxisFlange option on the revolute component of joint1 we can control the revolution as on Figure \ref{fig4}. Then we can provide a function that provides real value to the joint, this can be done with the Modelica blocks with a constant source block. A unit conversion block has to be used to convert from degrees to radians, there is a math block for that. Whith this done we provide the position for the revolute joint with a position rotational source. this will take the real input and will provide it as a signal that the joint can use. This 6 blocks represent the first joint link of the system as represented in Figure \ref{fig5}.

To set the system values we adjust the value on the joint parameters as shown on Figure \ref{fig6} and the joints values will change as shown on Figure \ref{fig7}

\begin{figure}
\centerline{\includegraphics[width=3.5in]{./images/Joint1Parameters}}
\caption{OpenModelica, Joint1 parameters\label{fig4}}
\end{figure}

\begin{figure}
\centerline{\includegraphics[width=3.5in]{./images/Joint1Link1+Controller}}
\caption{Joint1, Link1 and controller\label{fig5}}
\end{figure}



\begin{figure}
\centerline{\includegraphics[width=3.5in]{./images/theta1Parameters}}
\caption{Joint1 Parameters set to 45 $\deg$\label{fig6}}
\end{figure}

\begin{figure}
\centerline{\includegraphics[width=3.5in]{./images/Joint145Joint20}}
\caption{Simulation with Joint1 set to 45 $\deg$\label{fig7}}
\end{figure}

As we can see on Figure \ref{fig8} the values for the torque as the same as the Python programming and also the manual calculations.

\begin{figure}
\centerline{\includegraphics[width=3.5in]{./images/taus}}
\caption{Torques required for Joints 1 and 2 to move links 1 and 2\label{fig8}}
\end{figure}

An import aspect of exporting from Solidworks to OpenModelic is the compatibility, the exported file can be a $.STL$ file but the measurements have to be in meters and the object have to be modelled around the center of mass, also when importing the option "Do not translate STL output data to positive space" has to be checked to avoid misinterpretation between the softwares.

Your color graphic will be converted to grayscale if no separate grayscale 
file is provided. If a graphic is to appear in print as black and white, it 
should be saved and submitted as a black and white file. If a graphic is to 
appear in print or on IEEE Xplore in color, it should be submitted as RGB. The import parameters are shown on \ref{fig9} 

\begin{figure}
\centerline{\includegraphics[width=3.5in]{./images/exportSolidworks}}
\caption{Exporting parameters\label{fig9}}
\end{figure}

Then we can import the file into OpenModelica by refereing its center of mass and file location as shown on \ref{fig10}


\begin{figure}
\centerline{\includegraphics[width=3.5in]{./images/importOpenModelicaParameters}}
\caption{Importing parameters\label{fig10}}
\end{figure}


At Figure \ref{fig11} we can see the simulation model at OpenModelica.

\begin{figure*}
\centerline{\includegraphics[width=7in]{./images/Simulation}}
\caption{Simulation model, Base, rotational link and 2 more links\label{fig11}}
\end{figure*}

We can animate the simulation using OpenModelica as can be seen on Figure \ref{fig12}.

\begin{figure*}
\centerline{\includegraphics[width=7in]{./images/SimulationAnimation}}
\caption{Simulation animation\label{fig12}}
\end{figure*}

\section{Discussion and Conclusions}

The IEEE Graphics Checker Tool enables users to check graphic files. The 
tool will check journal article graphic files against a set of rules for 
compliance with IEEE requirements. These requirements are designed to ensure 
sufficient image quality so they will look acceptable in print. After 
receiving a graphic or a set of graphics, the tool will check the files 
against a set of rules. A report will then be e-mailed listing each graphic 
and whether it met or failed to meet the requirements. If the file fails, a 
description of why and instructions on how to correct the problem will be 
sent. The IEEE Graphics Checker Tool is available at \href 
{http://graphicsqc.ieee.org/}{http://graphicsqc.ieee.org/}

For more Information, contact the IEEE Graphics H-E-L-P Desk by e-mail at 
\href {mailto:graphics@ieee.org}{mailto:graphics@ieee.org}. You will then receive an e-mail response and 
sometimes a request for a sample graphic for us to check.


\begin{table}
\caption{Units for Magnetic Properties}
\label{table}
\setlength{\tabcolsep}{3pt}
\begin{tabular}{|p{25pt}|p{75pt}|p{115pt}|}
\hline
Symbol& 
Quantity& 
Conversion from Gaussian and \par CGS EMU to SI $^{\mathrm{a}}$ \\
\hline
$\Phi $& 
magnetic flux& 
1 Mx $\to 10^{-8}$ Wb $= 10^{-8}$ V$\cdot $s \\
$B$& 
magnetic flux density, \par magnetic induction& 
1 G $\to 10^{-4}$ T $= 10^{-4}$ Wb/m$^{2}$ \\
$H$& 
magnetic field strength& 
1 Oe $\to 10^{3}/(4\pi )$ A/m \\
$m$& 
magnetic moment& 
1 erg/G $=$ 1 emu \par $\to 10^{-3}$ A$\cdot $m$^{2} = 10^{-3}$ J/T \\
$M$& 
magnetization& 
1 erg/(G$\cdot $cm$^{3}) =$ 1 emu/cm$^{3}$ \par $\to 10^{3}$ A/m \\
4$\pi M$& 
magnetization& 
1 G $\to 10^{3}/(4\pi )$ A/m \\
$\sigma $& 
specific magnetization& 
1 erg/(G$\cdot $g) $=$ 1 emu/g $\to $ 1 A$\cdot $m$^{2}$/kg \\
$j$& 
magnetic dipole \par moment& 
1 erg/G $=$ 1 emu \par $\to 4\pi \times 10^{-10}$ Wb$\cdot $m \\
$J$& 
magnetic polarization& 
1 erg/(G$\cdot $cm$^{3}) =$ 1 emu/cm$^{3}$ \par $\to 4\pi \times 10^{-4}$ T \\
$\chi , \kappa $& 
susceptibility& 
1 $\to 4\pi $ \\
$\chi_{\rho} $& 
mass susceptibility& 
1 cm$^{3}$/g $\to 4\pi \times 10^{-3}$ m$^{3}$/kg \\
$\mu $& 
permeability& 
1 $\to 4\pi \times 10^{-7}$ H/m \par $= 4\pi \times 10^{-7}$ Wb/(A$\cdot $m) \\
$\mu_{r}$& 
relative permeability& 
$\mu \to \mu_{r}$ \\
$w, W$& 
energy density& 
1 erg/cm$^{3} \to 10^{-1}$ J/m$^{3}$ \\
$N, D$& 
demagnetizing factor& 
1 $\to 1/(4\pi )$ \\
\hline
\multicolumn{3}{p{251pt}}{Vertical lines are optional in tables. Statements that serve as captions for 
the entire table do not need footnote letters.} \\
\multicolumn{3}{p{251pt}}{$^{\mathrm{a}}$Gaussian units are the same as cg emu for magnetostatics; Mx 
$=$ maxwell, G $=$ gauss, Oe $=$ oersted; Wb $=$ weber, V $=$ volt, s $=$ 
second, T $=$ tesla, m $=$ meter, A $=$ ampere, J $=$ joule, kg $=$ 
kilogram, H $=$ henry.}
\end{tabular}
\label{tab2}
\end{table}

\subsection{Copyright Form}
An IEEE copyright form should accompany your final submission. You can get a 
.pdf, .html, or .doc version at \href 
{http://www.ieee.org/copyright}{http://www.ieee.org/copyright}. Authors are responsible for obtaining any 
security clearances.

\section{Units}
Use either SI (MKS) or CGS as primary units. (SI units are strongly 
encouraged.) English units may be used as secondary units (in parentheses). 
\textbf{This applies to papers in data storage.} For example, write ``15 
Gb/cm$^{2}$ (100 Gb/in$^{2})$.'' An exception is when English units are used 
as identifiers in trade, such as ``3$\frac12$-in disk drive.'' Avoid 
combining SI and CGS units, such as current in amperes and magnetic field in 
oersteds. This often leads to confusion because equations do not balance 
dimensionally. If you must use mixed units, clearly state the units for each 
quantity in an equation.

The SI unit for magnetic field strength $H$ is A/m. However, if you wish to use 
units of T, either refer to magnetic flux density $B$ or magnetic field 
strength symbolized as $\mu_{0}H$. Use the center dot to separate 
compound units, e.g., ``A$\cdot$m$^{2}$.''

\section{Helpful Hints}
\subsection{Figures and Tables}
Because IEEE will do the final formatting of your paper, you do not need to 
position figures and tables at the top and bottom of each column. Large 
figures and tables may span both columns. Place figure captions below the 
figures; place table titles above the tables. If your figure has two parts, 
include the labels ``(a)'' and ``(b)'' as part of the artwork. Please verify 
that the figures and tables you mention in the text actually exist. 
\textbf{Please do not include captions as part of the figures. Do not put 
captions in ``text boxes'' linked to the figures. Do not put borders around 
the outside of your figures.} Use the abbreviation ``Fig.'' even at the 
beginning of a sentence. Do not abbreviate ``Table.'' Tables are numbered 
with Roman numerals. 

Figure axis labels are often a source of confusion. Use words rather than 
symbols. As an example, write the quantity ``Magnetization,'' or 
``Magnetization $M$,'' not just ``$M$.'' Put units in parentheses. Do not label 
axes only with units. As in Fig. 1, for example, write ``Magnetization 
(A/m)'' or ``Magnetization (A$\cdot $m$^{-1})$,'' not just ``A/m.'' Do not 
label axes with a ratio of quantities and units. For example, write 
``Temperature (K),'' not ``Temperature/K.'' 

Multipliers can be especially confusing. Write ``Magnetization (kA/m)'' or 
``Magnetization (10$^{3}$ A/m).'' Do not write ``Magnetization (A/m) 
$\times$ 1000'' because the reader would not know whether the top axis 
label in Fig. 1 meant 16000 A/m or 0.016 A/m. Figure labels should be 
legible, approximately 8 to 12 point type.

\subsection{\LaTeX-Specific Advice}

Please use ``soft'' (e.g., \verb|\eqref{Eq}|) cross references instead
of ``hard'' references (e.g., \verb|(1)|). That will make it possible
to combine sections, add equations, or change the order of figures or
citations without having to go through the file line by line.

Please don't use the \verb|{eqnarray}| equation environment. Use
\verb|{align}| or \verb|{IEEEeqnarray}| instead. The \verb|{eqnarray}|
environment leaves unsightly spaces around relation symbols.

Please note that the \verb|{subequations}| environment in {\LaTeX}
will increment the main equation counter even when there are no
equation numbers displayed. If you forget that, you might write an
article in which the equation numbers skip from (17) to (20), causing
the copy editors to wonder if you've discovered a new method of
counting.

{\BibTeX} does not work by magic. It doesn't get the bibliographic
data from thin air but from .bib files. If you use {\BibTeX} to produce a
bibliography you must send the .bib files. 

{\LaTeX} can't read your mind. If you assign the same label to a
subsubsection and a table, you might find that Table I has been cross
referenced as Table IV-B3. 

{\LaTeX} does not have precognitive abilities. If you put a
\verb|\label| command before the command that updates the counter it's
supposed to be using, the label will pick up the last counter to be
cross referenced instead. In particular, a \verb|\label| command
should not go before the caption of a figure or a table.

Do not use \verb|\nonumber| or \verb|\notag| inside the \verb|{array}| environment. It
will not stop equation numbers inside \verb|{array}| (there won't be
any anyway) and it might stop a wanted equation number in the
surrounding equation.

\subsection{References}
Number citations consecutively in square brackets \cite{ref1}. The sentence 
punctuation follows the brackets \cite{ref2}. Multiple references \cite{ref2}, \cite{ref3} are each 
numbered with separate brackets \cite{ref1}--\cite{ref3}. When citing a section in a book, 
please give the relevant page numbers \cite{ref2}. In sentences, refer simply to the 
reference number, as in \cite{ref3}. Do not use ``Ref. \cite{ref3}'' or ``reference \cite{ref3}'' 
except at the beginning of a sentence: ``Reference \cite{ref3} shows $\ldots$ .'' Please 
do not use automatic endnotes in \emph{Word}, rather, type the reference list at the 
end of the paper using the ``References'' style.

Number footnotes separately in superscripts (Insert $\vert$ 
Footnote).\footnote{It is recommended that footnotes be avoided (except for 
the unnumbered footnote with the receipt date on the first page). Instead, 
try to integrate the footnote information into the text.} Place the actual 
footnote at the bottom of the column in which it is cited; do not put 
footnotes in the reference list (endnotes). Use letters for table footnotes 
(see Table I). 

Please note that the references at the end of this document are in the 
preferred referencing style. Give all authors' names; do not use ``\emph{et al}.'' 
unless there are six authors or more. Use a space after authors' initials. 
Papers that have not been published should be cited as ``unpublished'' \cite{ref4}. 
Papers that have been accepted for publication, but not yet specified for an 
issue should be cited as ``to be published'' \cite{ref5}. Papers that have been 
submitted for publication should be cited as ``submitted for publication'' 
\cite{ref6}. Please give affiliations and addresses for private communications \cite{ref7}.

Capitalize only the first word in a paper title, except for proper nouns and 
element symbols. For papers published in translation journals, please give 
the English citation first, followed by the original foreign-language 
citation \cite{ref8}.

\subsection{Abbreviations and Acronyms}
Define abbreviations and acronyms the first time they are used in the text, 
even after they have already been defined in the abstract. Abbreviations 
such as IEEE, SI, ac, and dc do not have to be defined. Abbreviations that 
incorporate periods should not have spaces: write ``C.N.R.S.,'' not ``C. N. 
R. S.'' Do not use abbreviations in the title unless they are unavoidable 
(for example, ``IEEE'' in the title of this article).

\subsection{Equations}
Number equations consecutively with equation numbers in parentheses flush 
with the right margin, as in (\ref{eq1}). First use the equation editor to create 
the equation. Then select the ``Equation'' markup style. Press the tab key 
and write the equation number in parentheses. To make your equations more 
compact, you may use the solidus ( / ), the exp function, or appropriate 
exponents. Use parentheses to avoid ambiguities in denominators. Punctuate 
equations when they are part of a sentence, as in
\begin{multline}
\int_{0}^{r_{2}} {F(r,\phi )} \,dr\,d\phi =[\sigma r_{2} /(2\mu_{0} )] \\ 
 \cdot \int_{0}^{\infty} \exp (-\lambda \vert 
z_{j} -z_{i} \vert )\lambda^{-1}J_{1} (\lambda r_{2} )J_{0} 
(\lambda r_{i} )\,d\lambda . 
\label{eq1}
\end{multline}
Be sure that the symbols in your equation have been defined before the 
equation appears or immediately following. Italicize symbols ($T$ might refer 
to temperature, but T is the unit tesla). Refer to ``(\ref{eq1}),'' not ``Eq. (\ref{eq1})'' 
or ``equation (\ref{eq1}),'' except at the beginning of a sentence: ``Equation (\ref{eq1}) 
is $\ldots$ .''

\subsection{Other Recommendations}
Use one space after periods and colons. Hyphenate complex modifiers: 
``zero-field-cooled magnetization.'' Avoid dangling participles, such as, 
``Using (\ref{eq1}), the potential was calculated.'' [It is not clear who or what 
used (\ref{eq1}).] Write instead, ``The potential was calculated by using (\ref{eq1}),'' or 
``Using (\ref{eq1}), we calculated the potential.''

Use a zero before decimal points: ``0.25,'' not ``.25.'' Use ``cm$^{3}$,'' 
not ``cc.'' Indicate sample dimensions as ``0.1 cm~$\times$~0.2 cm,'' not 
``0.1~$\times$~0.2 cm$^{2}$.'' The abbreviation for ``seconds'' is ``s,'' 
not ``sec.'' Do not mix complete spellings and abbreviations of units: use 
``Wb/m$^{2}$'' or ``webers per square meter,'' not ``webers/m$^{2}$.'' When 
expressing a range of values, write ``7 to 9'' or ``7-9,'' not 
``7$\sim$9.''

A parenthetical statement at the end of a sentence is punctuated outside of 
the closing parenthesis (like this). (A parenthetical sentence is punctuated 
within the parentheses.) In American English, periods and commas are within 
quotation marks, like ``this period.'' Other punctuation is ``outside''! 
Avoid contractions; for example, write ``do not'' instead of ``don't.'' The 
serial comma is preferred: ``A, B, and C'' instead of ``A, B and C.''

If you wish, you may write in the first person singular or plural and use 
the active voice (``I observed that $\ldots$'' or ``We observed that $\ldots$'' 
instead of ``It was observed that $\ldots$''). Remember to check spelling. If 
your native language is not English, please get a native English-speaking 
colleague to carefully proofread your paper.

\section{Some Common Mistakes}
The word ``data'' is plural, not singular. The subscript for the 
permeability of vacuum $\mu_{0}$ is zero, not a lowercase letter ``o.'' 
The term for residual magnetization is ``remanence''; the adjective is 
``remanent''; do not write ``remnance'' or ``remnant.'' Use the word 
``micrometer'' instead of ``micron.'' A graph within a graph is an 
``inset,'' not an ``insert.'' The word ``alternatively'' is preferred to the 
word ``alternately'' (unless you really mean something that alternates). Use 
the word ``whereas'' instead of ``while'' (unless you are referring to 
simultaneous events). Do not use the word ``essentially'' to mean 
``approximately'' or ``effectively.'' Do not use the word ``issue'' as a 
euphemism for ``problem.'' When compositions are not specified, separate 
chemical symbols by en-dashes; for example, ``NiMn'' indicates the 
intermetallic compound Ni$_{0.5}$Mn$_{0.5}$ whereas ``Ni--Mn'' indicates an 
alloy of some composition Ni$_{x}$Mn$_{1-x}$.

Be aware of the different meanings of the homophones ``affect'' (usually a 
verb) and ``effect'' (usually a noun), ``complement'' and ``compliment,'' 
``discreet'' and ``discrete,'' ``principal'' (e.g., ``principal 
investigator'') and ``principle'' (e.g., ``principle of measurement''). Do 
not confuse ``imply'' and ``infer.'' 

Prefixes such as ``non,'' ``sub,'' ``micro,'' ``multi,'' and ``ultra'' are 
not independent words; they should be joined to the words they modify, 
usually without a hyphen. There is no period after the ``et'' in the Latin 
abbreviation ``\emph{et al.}'' (it is also italicized). The abbreviation ``i.e.,'' means 
``that is,'' and the abbreviation ``e.g.,'' means ``for example'' (these 
abbreviations are not italicized).

An excellent style manual and source of information for science writers is 
\cite{ref9}. A general IEEE style guide and an \emph{Information for Authors} are both available at \href 
{http://www.ieee.org/web/publications/authors/transjnl/index.html}{http://www.ieee.org/web/publications/authors/transjnl/index.html}

\section{Editorial Policy}
Submission of a manuscript is not required for participation in a 
conference. Do not submit a reworked version of a paper you have submitted 
or published elsewhere. Do not publish ``preliminary'' data or results. The 
submitting author is responsible for obtaining agreement of all coauthors 
and any consent required from sponsors before submitting a paper. IEEE 
TRANSACTIONS and JOURNALS strongly discourage courtesy authorship. It is the 
obligation of the authors to cite relevant prior work.

The Transactions and Journals Department does not publish conference records 
or proceedings. The TRANSACTIONS does publish papers related to conferences 
that have been recommended for publication on the basis of peer review. As a 
matter of convenience and service to the technical community, these topical 
papers are collected and published in one issue of the TRANSACTIONS.

At least two reviews are required for every paper submitted. For 
conference-related papers, the decision to accept or reject a paper is made 
by the conference editors and publications committee; the recommendations of 
the referees are advisory only. Undecipherable English is a valid reason for 
rejection. Authors of rejected papers may revise and resubmit them to the 
TRANSACTIONS as regular papers, whereupon they will be reviewed by two new 
referees.

\section{Publication Principles}
The contents of IEEE TRANSACTIONS and JOURNALS are peer-reviewed and 
archival. The TRANSACTIONS publishes scholarly articles of archival value as 
well as tutorial expositions and critical reviews of classical subjects and 
topics of current interest. 

Authors should consider the following points:

\begin{enumerate}
\item Technical papers submitted for publication must advance the state of knowledge and must cite relevant prior work. 
\item The length of a submitted paper should be commensurate with the importance, or appropriate to the complexity, of the work. For example, an obvious extension of previously published work might not be appropriate for publication or might be adequately treated in just a few pages.
\item Authors must convince both peer reviewers and the editors of the scientific and technical merit of a paper; the standards of proof are higher when extraordinary or unexpected results are reported. 
\item Because replication is required for scientific progress, papers submitted for publication must provide sufficient information to allow readers to perform similar experiments or calculations and use the reported results. Although not everything need be disclosed, a paper must contain new, useable, and fully described information. For example, a specimen's chemical composition need not be reported if the main purpose of a paper is to introduce a new measurement technique. Authors should expect to be challenged by reviewers if the results are not supported by adequate data and critical details.
\item Papers that describe ongoing work or announce the latest technical achievement, which are suitable for presentation at a professional conference, may not be appropriate for publication in a TRANSACTIONS or JOURNAL.
\end{enumerate}

\section{Conclusion}
Please include a brief summary of the possible clinical implications of your 
work in the conclusion section. Although a conclusion may review the main 
points of the paper, do not replicate the abstract as the conclusion. 
Consider elaborating on the translational importance of the work or suggest 
applications and extensions. 


\clearpage
\newpage

\begin{thebibliography}{00}
\bibitem{ref1} J. J. Craig, \emph{Introduction To Robotics: Mechanics And Control}, 3rd ed., Ed. New York: Pearson Education, 2009.
\bibitem{ref2} P. Fritzson, et al. \emph{The OpenModelica Integrated Modeling, Simulation and Optimization Environment} (Conference paper), PROCEEDINGS OF THE 1ST AMERICAN MODELICA CONFERENCE$. $Cambridge, USA: Massechusetts, 2018.
\bibitem{ref3} W. Bolton, \emph{Mechatronics - Electronic control systems in mechanical and electrical engineering}. United Kingdom: Pearson Education Limited, 2019.
\bibitem{ref4} B. Smith, ``An approach to graphs of linear forms (Unpublished work style),'' unpublished.
\bibitem{ref5} E. H. Miller, ``A note on reflector arrays (Periodical style---Accepted for publication),'' \emph{IEEE Trans. Antennas Propagat.}, to be published.
\bibitem{ref6} J. Wang, ``Fundamentals of erbium-doped fiber amplifiers arrays (Periodical style---Submitted for publication),'' \emph{IEEE J. Quantum Electron.}, submitted for publication.
\bibitem{ref7} C. J. Kaufman, Rocky Mountain Research Lab., Boulder, CO, private communication, May 1995.
\bibitem{ref8} Y. Yorozu, M. Hirano, K. Oka, and Y. Tagawa, ``Electron spectroscopy studies on magneto-optical media and plastic substrate interfaces (Translation Journals style),'' \emph{IEEE Transl. J. Magn.Jpn.}, vol. 2, Aug. 1987, pp. 740--741 [\emph{Dig. 9}$^{th}$\emph{ Annu. Conf. Magnetics} Japan, 1982, p. 301].
\bibitem{ref9} M. Young, \emph{The Technical Writers Handbook.} Mill Valley, CA: University Science, 1989.
\bibitem{ref10} J. U. Duncombe, ``Infrared navigation---Part I: An assessment of feasibility (Periodical style),'' \emph{IEEE Trans. Electron Devices}, vol. ED-11, pp. 34--39, Jan. 1959.
\bibitem{ref11} S. Chen, B. Mulgrew, and P. M. Grant, ``A clustering technique for digital communications channel equalization using radial basis function networks,'' \emph{IEEE Trans. Neural Networks}, vol. 4, pp. 570--578, Jul. 1993.
\bibitem{ref12} R. W. Lucky, ``Automatic equalization for digital communication,'' \emph{Bell Syst. Tech. J.}, vol. 44, no. 4, pp. 547--588, Apr. 1965.
\bibitem{ref13} S. P. Bingulac, ``On the compatibility of adaptive controllers (Published Conference Proceedings style),'' in \emph{Proc. 4th Annu. Allerton Conf. Circuits and Systems Theory}, New York, 1994, pp. 8--16.
\bibitem{ref14} G. R. Faulhaber, ``Design of service systems with priority reservation,'' in \emph{Conf. Rec. 1995 IEEE Int. Conf. Communications,} pp. 3--8.
\bibitem{ref15} W. D. Doyle, ``Magnetization reversal in films with biaxial anisotropy,'' in \emph{1987 Proc. INTERMAG Conf.}, pp. 2.2-1--2.2-6.
\bibitem{ref16} G. W. Juette and L. E. Zeffanella, ``Radio noise currents n short sections on bundle conductors (Presented Conference Paper style),'' presented at the IEEE Summer power Meeting, Dallas, TX, Jun. 22--27, 1990, Paper 90 SM 690-0 PWRS.
\bibitem{ref17} J. G. Kreifeldt, ``An analysis of surface-detected EMG as an amplitude-modulated noise,'' presented at the 1989 Int. Conf. Medicine and Biological Engineering, Chicago, IL.
\bibitem{ref18} J. Williams, ``Narrow-band analyzer (Thesis or Dissertation style),'' Ph.D. dissertation, Dept. Elect. Eng., Harvard Univ., Cambridge, MA, 1993. 
\bibitem{ref19} N. Kawasaki, ``Parametric study of thermal and chemical nonequilibrium nozzle flow,'' M.S. thesis, Dept. Electron. Eng., Osaka Univ., Osaka, Japan, 1993.
\bibitem{ref20} J. P. Wilkinson, ``Nonlinear resonant circuit devices (Patent style),'' U.S. Patent 3 624 12, July 16, 1990. 
\bibitem{ref21} \emph{IEEE Criteria for Class IE Electric Systems} (Standards style)$,$ IEEE Standard 308, 1969.
\bibitem{ref22} \emph{Letter Symbols for Quantities}, ANSI Standard Y10.5-1968.
\bibitem{ref23} R. E. Haskell and C. T. Case, ``Transient signal propagation in lossless isotropic plasmas (Report style),'' USAF Cambridge Res. Lab., Cambridge, MA Rep. ARCRL-66-234 (II), 1994, vol. 2.
\bibitem{ref24} E. E. Reber, R. L. Michell, and C. J. Carter, ``Oxygen absorption in the Earth's atmosphere,'' Aerospace Corp., Los Angeles, CA, Tech. Rep. TR-0200 (420-46)-3, Nov. 1988.
\bibitem{ref25} (Handbook style) \emph{Transmission Systems for Communications,} 3rd ed., Western Electric Co., Winston-Salem, NC, 1985, pp. 44--60.
\bibitem{ref26} \emph{Motorola Semiconductor Data Manual,} Motorola Semiconductor Products Inc., Phoenix, AZ, 1989.
\bibitem{ref27} (Basic Book/Monograph Online Sources) J. K. Author. (year, month, day). \emph{Title} (edition) [Type of medium]. Volume (issue). Available: \underline {http://www.(URL})
\bibitem{ref28} J. Jones. (1991, May 10). Networks (2nd ed.) [Online]. Available: \underline {http://www.atm.com}
\bibitem{ref29} (Journal Online Sources style) K. Author. (year, month). Title. \emph{Journal} [Type of medium]. Volume(issue), paging if given. Available: \underline {http://www.(URL})
\bibitem{ref30} R. J. Vidmar. (1992, August). On the use of atmospheric plasmas as electromagnetic reflectors. \emph{IEEE Trans. Plasma Sci.} [Online]. \emph{21(3).} pp. 876--880. Available: http://www.halcyon.com/pub/journals/21ps03-vidmar
\end{thebibliography}



\clearpage
\newpage


\appendix
\label{appendix-A}



\includepdf[pages=-, frame=false, fitpaper=true, lastpage=8, scale=1, pagecommand={\thispagestyle{plain}}]{./images/MatricesManipulation.pdf}



\end{document}
